\documentclass[12pt,a4paper]{book}
\usepackage[utf8]{inputenc}
\usepackage{amsmath}
\usepackage{amsfonts}
\usepackage{amssymb}
\usepackage[english]{babel}
\usepackage[numbers]{natbib}
\usepackage{graphicx}
\usepackage{float}

\title{Uncovering the properties of triple junctions}
\author{Paul Twine}
\begin{document}
\maketitle

\chapter{Literature Review}

\nocite{*}
A fundamental question in material science is how to find relationships between the microstructure and the macroscopic properties of a material. Polycrystalline materials comprise of grains, grain boundaries and triple junctions and their properties are essential to understanding the microstructure. The grain boundary is a two dimensional interface between two misorientated grains. A junction line occurs where three or more grain boundaries meet. The most common junction line is a triple line which has three adjacent grains. This paper will focus primarily on triple line junctions. Other higher order stable junction types do exist and a quad line is also investigated analytically in this section. 

Grain boundaries always have a positive excess energy compared to a pristine crystal at zero kelvin. In general the excess energy of a planar grain boundary depends upon the relative misorientation between the two adjacent crystals and also the normal direction of the planer grain boundary. The macroscopic relationship between two planar grain boundaries can be described by five degrees of freedom. 

The five degrees of freedom can be clearly demonstrated by considering the relative orientations of the two crystals and the normal direction of the planar grain boundary. Firstly the misorientation between two crystals can be defined by an axis of rotation and an angle of rotation about that axis.  The axis only defines a direction and can be completely specified by two degrees of freedom. A third degree of freedom is then the angle of rotation about that axis. Finally the grain boundary itself is a plane which can be defined by a normal unit vector which requires a further two degrees of freedom as there is no need to distinguish between antiparallel and parallel vectors.


For non-planar grain boundaries an increase curvature also generally increases the excess energy.  The excess energy in a grain boundary can be lowered considerably if they two lattices of the neighbouring grains form a coincident site lattice. As some atoms within the grain boundary retain some of their atomic bonding they have a lower overall potential energy due to a higher cohesive energy. Whilst grains and grain boundaries have been studied extensively there has been relatively little work on triple lines.

In general the excess energy associated with a grain bounary will vary periodically along the length of the grain boundary. This is to expected as is defined as the intersection of two periodic grain structures. The periodicity may be obscured if the length of the grain boundary is less than the periodic length. An intutive measure of the grain boundary energy is to average the excess energy over one period and then normalise this per unit of surface area.

A general method already exists for finding the excess energy associated with a planar grain boundary of a bicrystal. Firstly the potential energy of a single crystal with a fixed number of atoms is calculated. Then this crystal is bisected and the orientation of one half is changed so that there exists a crystallographic misorientation and a planar interface between the two halves of the crystal. The difference in the energy between the bisected crystal and the original single crystal can then be taken to be the excess energy of the grain boundary.

To identify the excess energy associated with a triple line is more complicated and there is no obvious analogous method to the grain boundary approach of the previous paragraph. The specific issues is that a triple line cannot exist without three grain boundaries. However a similar scenario involving quad line can provide some phyiscial intuition and a possible approach to find the formation energy of a quad line. 

\begin{figure}[H]
	\centering
	\includegraphics[scale=0.5]{QuadConfigurations.png} 
\label{fig:1}
\caption{Two periodic cells showing a simplified formation process of a quad line.}
\end{figure}

Figure \ref{fig:1}
 shows the analogous situation to the excess energy grain boundary derivation using two different configurations. This is a plan view of initial configuration of three grains separated by three grain boundaries. The grains are labelled $1, 2$ and $3$ and the grain boundaries are coloured red, green and blue. This is a periodic cell which is repeated in the vertical and horizonal directions. Each grain is also extruded out of the page and so the system is pseudo two dimensional.
 
In the periodic cell on the left there already exists one periodically unique triple line which is formed at the four corners of the rectangle. The grain boundaries shown in blue and green do not intersect near the middle of the cell. The initial configuration is then a set of three grains and three grain boundaries.

The periodic cell shown on the right is a modification of the cell on the left and contains an indentical number of atoms. The only change is that grain boundary in blue has move upwards so that the green and blue boundaries intersect. This now a system of three grains and four grain boundaries as the blue grain boundary has now been bisected. The junction in the middle is now a quad line.

The movement of the blue grain boundary does not affect its normal  or the orientations of grains $B$ and $C$ and so it is reasonable to expect that the interfacial energy of the blue grain boundary does not change significantly. There is also an implicit assumption that the shape of the grain boundaries do not distort as a result of forming the quad line. A lower bound for the formation energy of a quad line can now be estimated by considering the blue, red, green and yellow regions of defective atoms.

The intersecting region of the blue and green grain boundaries is shown in yellow. The total number of atoms is fixed and this overlapping region reduces the number of defective of atoms in the grain boundaries. Therefore it also increases the number of lattice atoms in the bulk. Provided that the energy increase in the yellow region is less than the enery reduction due to the atoms migrating from the grain boundary to the lattice the triple line will have a negative formation energy. 
 
The lower bound for energy of formation of the quad line can be estimated by comparing the two periodic cells. Firstly using a mass balance gives the following
equation

\[ n_1 + n_2 + n_3 + n_B + n_G + n_R = n_1^{\prime} + n_2^{\prime} + n_3^{\prime} + n_B^{\prime} + n_G^{\prime} + n_R^{\prime} +
n_Y^{\prime} \]

where $n_i$ is the number of atoms in lattice $i$ and $n_B, n_G$ and $n_R$ are the number of atoms in the blue, green and red grain boundaries respectively. The $\prime$ indicates the values for the right hand periodic cell after the blue grain boundary has moved.  The number of atoms in the intersecting yellow region is given by $n_Y^{\prime}$.

The equation can be simplified by assuming that $n_G^{\prime} = n_G-n_Y^{\prime}$, $n_B^{\prime} = n_B-n_Y^{\prime}$ and $n_R^{\prime} = n_R$. 

\[ n_1 + n_2 + n_3   = n_1^{\prime} + n_2^{\prime} + n_3^{\prime}- n_Y^{\prime} \]

A key result here is that the the final number of lattice atoms is larger by an amount $n_Y^{\prime}$ than the inital number of lattice atoms. The difference in energy between the two periodic cells gives an lower bound estimate of the formation energy $E_f$ of the quad line.

\begin{align*}
 E_{fl} &= U_L(n_1+n_2+n_3+n_Y^{\prime}) + U_G(n_G - n_Y^{\prime})
+ U_B(n_B - n_Y^{\prime}) \\ 
&+ U_R n_R + U_Q n_Y^{\prime}
- U_L(n_1+n_2+n_3) - U_G n_G - U_B n_B - U_R n_R  
\\ 
&=  n_Y^{\prime}(U_L - U_G - U_B + U_Q) = n_Y^{\prime}((U_L - U_G) - (U_B - U_Q)) 
\end{align*}

where $U_L, U_G$, $U_B$ and $U_Q$ are the mean potential energies per atom in the lattice, green grain boundary, blue grain boundary
and quad line respectively. Note that the final factorisation shows that provided that $U_L < U_G$ and $U_Q > U_B$ the formation energy will be negative.

Defining $U_M = (U_B + U_G)/2$ as the mean grain boundary energy per atom averaged over the blue and green grain boundaries then
\[E_{fl} =  n_Y^{\prime}(U_L + U_Q - 2U_M)\]

An upper bound for the energy of formation of the quad line can also be estimated by considering a slight modification to the above argument. If it is assumed that the total number of defective atoms remains constant and hence so does the number of lattice atoms then the energy of formation of the quad line can
be written as

\[E_{fu} =  n_Y^{\prime}(U_Q - U_M) = E_{fl} + n_Y^{\prime}(U_M - U_L) \]


 
Triple lines have been investigated as purely mathematical objects and a good summary of the key points can be found in \cite{Taylor1999}. The discussion includes scenarios where the triple line can have positive and negative formation energies. A key feature here is whether a triple line is considered to be a purely geometric line or whether it has a non-zero volume and consists of atoms. Taylor discusses different scenarios and specifically addresses potential issues with negative triple lines. Taylor argues that extra cylindrical sections could be added to a planar grain boundary so that the axis of symmetry is perpendicular to the grain boundary normal. In this way extra triple lines could be added and if they had a negative excess energy then the energy of the system could be made increasingly more negative. However such a process necessarily also required extra grain boundary to be created which has a positive excess energy and so this process does not seem physically plausible when viewed from an atomistic perspective.


The energy content of triple lines has been investigated by Srinivasan et al in \cite{Srinivasan1999} using atomistic simulation. Here there is an interesting discussion of the apparent paradox of the high potential energy expected in a triple line and a possible negative formation energy. 

Srinivasan also mentions that the triple line can be described using eleven degrees of freedom. Although this assumes it is the intersection of three planar grain boundaries. By assigning a coordinate system to be aligned with one of the three grains there are three degrees of freedom for each of the other two grains which can be orientated independently of each other. Two of the grain boundaries can then be orientated independently of each other giving a further two degrees of freedom each. However these two grain boundaries will in general intersect along a line if the pathological cases of identical grain boundary planes and parallel grain boundary planes are excluded. Therefore the final grain boundary has only one degree of freedom which can be interpreted as an angle of rotation of the grain boundary around the triple line.

The eleven degrees of freedom of a triple line are a result of applying a constraint to the geometry and in general three arbitrary planes cannot be made to intersect along a single line. Clearly triple lines can contribute vastly to the possible parameter space for a given simulation and only a small subset of all the possible angles can be considered. In real materials grain boundaries are curved and each grain boundary and triple line has a non-zero volume. 

 The results of the research appeared to showed a negative excess energy associated with the triple line. However the approach computed a mean average of several triple lines and this does not guarantee that every triple line in the simulation cell had a negative excess energy.The work also only considered a few specific grain configurations with relatively simple misorientation angles. This papert will extend this approach by considered a wider set of grain configurations and also seek to compute the energies of individual triple lines.

Theoretical work by Schvindlerman looked at the effect that triplelines could have on the mobility of the adjacent grain boundaries. A key result from the paper suggests that there is a critical grain radius below which that the driving force due to the excess energy of the triple line is more significant than the driving force due to the excess energy from the grain boundaries. This is a substantial finding which demonstrates that in smaller grains the effect of the triple line cannot be ignored. It is also significant that the paper suggest their is effective range of a triple line.  "Thermodynamics and kinetics of grain boundary triple junctions in metals: Recent developments"

A key feature of triples lines is described by the Herring relationship. The angles between three planar grain boundaries that meet at a triple line is interpreted using an energy minimisation. The analysis focusses on a small region surrounding
the triple line and is a first order approximation. As such the curvature of the grain boundaries is ignored and the triple line itself has zero volume.

The Herring relationship can be understood as minimising the energy in creating the three planar grain boundary interfaces. The expression is essentially a force balance where the force is an energy gradient associated with \emph{creating} the interface. This is often described as a line tension associated with the grain boundary but does not refer to any type of elastic deformation of the lattice. and torque term which act in perpendicular to the grain boundary normal and parallel to the grain boundary normal respectively. The grain boundaries and triple lines are assumed to be purely mathematical planes and lines with no associated volume. The energy in the system is also associated entirely with the grain boundaries and the triple line is position to minimise this energy. 

\[\gamma_{A/B}\mathbf{\hat{n}}_{A/B} + \gamma_{B/C}\mathbf{\hat{n}}_{B/C} + \gamma_{A/C}\mathbf{\hat{n}}_{A/C} =\mathbf{0} \]

This is an equilibrium condition and states that the angles of the adjacent grain boundaries and the position of the triple line will minimise the \emph{local} potential energy of the system. The Herring relationship is a useful condition and although it assumes the system can be accurately described by continuous variables. This may not be accurate as it is not consistent with a lattice based atomistic model where within the lattice atoms can only occupy discrete positions.     

An increasing amount of research suggests that triple lines may affect how  grains deform and also serve as a diffusion pipe. The stability of polycrystalline structure depends upon the potential energy associated with each region within the grain. The creation of an interface between the two grains has an corresponding excess energy compared to the pristine lattice at zero kelvin. However the energy associated with a triple line remains an open question which is investigated in this paper.

As the scope of the paper is restricted to pure metals the embedded atom method has been chosen as the theoretical basis of the emperical potentials. The embedded atom method is considered to be superior to a simple pair wise potential especially when simulating defects. As the main focus is the behaviour of triples lines and their neighbouring grain boundaries it is particularly important that the emperical potential can model defects accurately. "The embedded-atom method:
a review of theory and applications" This includes a discussion of how accurately the embedded atom method models grain boundaries when compared  experimental results.

There author could find little work that connects the theory of line dislocations with the triple line junction. Yet atomistic simulation of a variety of grain configurations shows that there is a long range effect of a triple line which can be modelled by adapting the mathematical approach used to describe the triple line.  

The paper is structured as follows:

 


\chapter{Molecular Dynamics}
The triple junction can be investigated using molecular dynamics simulation in the LAMMPS package. The periodic hexagonal configuration is constructed for a variety of misorientation angles which affect the excess interfacial energy at each grain boundary. The potential energy of grain boundaries and triple lines can also be estimated using an emperical potential by associating regions inside the simulation cell with the triple lines and grain boundaries.

The simulations cell is periodic and a plan view of the repeating hexagonal grid is shown in Figure $\ref{fig:PerCell}$. The black rectangle is the boundary of the simulation cell. 
\begin{figure}[H] \label{fig:PerCell}
	\includegraphics[scale=0.75]{GrainConfigurationPeriodic.png} 
\end{figure}
All the grain boundaries are tilts boundaries as the rotation axis is parallel to the simulation cell $\mathbf{k}$ vector which points vertically upwards and hence perpendicular to the normal vector of the vertical planar grain boundaries. The simulation cell consists of three hexgonal grains in a periodic configuration as shown below.

Although there are only three grains this produces six periodically distinct triple lines and nine periodically distinct grain boundaries. The grain boundary plane as well as the misorientation between the adjacent grains both contribute to the excess interfacial energy of the grain boundary and so in general each grain boundary will have a different excess energy.

The simulation in LAMMPS begins with a conjugate gradient energy minimisation to relax the grain structure. This is followed by a $300$ Kelvin thermal annealing using a canonical ensemble. This ensures that the number of particles, the volume of the simulation cell and the temperature are all conserved.  Even at $300$ Kelvin there was a small amount of melting shown by some non-PTM atoms appearing in the bulk. 

\includegraphics[scale=1]{CellWithMelting.png} 

It is also apparent that even with a very low choice of RMSD as $0.05$ the non-PTM atoms do not from continuously connected grain boundaries between the triple lines. 


\subsection{Grain Boundary and Triple Line Descriptors}

The constituent parts of a simulation cell comprise of the grains, grain boundaries and triple lines. The use of an emperical potential allows for different regions of space to be assigned a proportion of the supercell's potential energy. In order to do this an algorithm is required to identify regions that correspond to grain boundaries, triple lines and grain atoms.

As an initial first step the polyhedral template matching alogrithm (PTM) is used to make a binary choice as to whether an atom is a PTM or non-PTM atom. The atoms are matched to an FCC structure with a lattice parameter of $a=4.05$. If an atom is indentified as belonging to a different crystallographic structure then this is also designated as a non-PTM atom. The majority of the PTM atoms will reside in the bulk and similarly the majority of non-PTM atoms will be considered grain boundary or triple line atoms. 

It is important to emphasise that PTM atoms \emph{do not} define whether an atom is part of a grain or in the grain boundary. In fact it is desirable to define grain boundaries with different widths as discussed in section LABEL here. The PTM atoms are only used as a starting point for defining triple line positions and grain boundaries.

The process of triple line detection uses an image analysis technique based upon the positions of the non-PTM atoms. The process begins by projecting all the non-PTM atoms onto the $xy$ plane of the simulation cell. The $xy$ plane is then partioned into a grid consisting of parallelograms. The side lengths of each parallelograms are chosen to be equal to the lattice parameter. Each paralleogram is then shaded if there are more than a specified number of non-PTM atoms whose projected co-ordinates lie inside the parallelogram. This minimum number is parameter that can be chosen and depends upon the height of the simulation cell.

The shading of the parallelograms is to give an indication of where the grain boundaries lies and their intersections will also give the approximate position of the triple lines. In general the base of the simulation cell will be a parallelogram however it is easier to deal with the periodicity of the simulation cell by transforming the points into a rectangular parameter space. 

In order to achieve a continuous shape several image processing techniques are used. Firstly the small holes within the image are filled with a default setting of an area of less than or equal to four pixels. Then the image is also dilated so that each yellow point also has its neighbouring points with in a fixed distance coloured yellow. This has the effect of filling in and broading the pixelated represention of the grain boundaries and is shown in the first diagram on the left of Figure $\ref{fig:ImSt}$. 

The value chosen for the dilation is a compromise between ensuring their are now gaps in the grain boundaries and trying to retain as much of the detail as possbible in the grain boundary shapes and positions. Small objects with an area of 4 pixels or less are also removed. This is to remove small groups of defective atoms that may appear in the bulk but are not part of a grain boundary or triple line. At this stage the edges of the yellow shaded regions are not generally smooth.

 
\begin{figure}[H]
	\centering
	\includegraphics[scale=0.4]{ImageSteps.png}   
\end{figure}
\label{fig:ImSt}

The periodicity in the plane of the base of the simulation cell is included by extending the shape so that each triple line that lies near the boundary of the simulation cell can also be detected. The yellow region has also been smoothed using a Gaussian filter. The results are shown in the middel diagram of Figure $\ref{fig:ImSt}$. The scale of the vertical and horizontal axes have changed and . 

Finally the image is then reduced to a skeleton of points. The skeleton preserves the same connectivity of the original image but reduces the image to a series of lines each with a width of one pixel. The skeleton image is then used to detect triple lines and also form an adjacency matrix for the triple lines. The third image on the right of Figure $\ref{fig:ImSt}$ shows the skeleton image and the triple lines are now yellow dots. The grain boundaries and grain boundary fragments are now approximated by the light blue lines.

The diagram on the right Figure $\ref{fig:ImSt}$ also includes green dots  where a grain boundary appears to terminate in the bulk. These points are \emph{not} triple lines and a blue line connecting a yellow triple line to a green dot is \emph{not} considered a grain boundary. In fact these are periodic repeat of part of an grain boundary that has already been included inside the periodic cell and can be discarded from the remaining analysis.

\begin{itemize}

	\item LAMMPS simulation data is categorised using the Ovito PTM algorithm. This classifies atoms based upon their lattice type.
	\item Atoms that are not grain  atoms have their coordinates quantised so that the occur in a two dimensional square grid whose side length is the particular metal lattice parameter
	\item The associated grid image is then reduced in such a way that the grain boundary images all have width of only one grid whilst preserving the connectivity of the structure.
	\item A connectivity algorithm is then run to find grain boundary points that have three neighbouring grain boundary points.
	\item In general these will be triple lines although there are a couple of pathological examples where this fails.
	\item The triple points are then deleted leaving a series of disconneted line shapes that make up the grain boundaries.
	\item Each geometric description of the grain boundary or triple line is scaled back into the simulation cell. 
	\item Finally each positional point is refined by using a spherical volume centred at the point. The mean position of the nongrain atoms within the sphere is then defined to be a point that lines upon the triple line. 

\end{itemize} 


\section{Statistical Analysis of Potential Energy}

A statistical analysis of the spatial variation of potential energy can show how the potential energy is partioned. Consider an axially symmetric region with the triple line as the axis of symmetry. For a small radius the potential energy inside the cylinder does not contain any bulk atoms. As the radius increases the potential energy will include contributions from the three neighbouring grain boundaries and also the lattice atoms. The potential energy of an atom in the perfect lattice is defined as part of the emperical potential. Furthermore the potential energy contribution from each grain boundary is expected to increase approximately linearly with grain boundary length.

This approach has already been employed in \citep{Srinivasan1999}   
and produced statistical averages for the potential energies of the triple lines and grain boundaries. An inherenet difficulty with this approach is the periodic variation of potential energy along the grain boundaries produces a lot of statistical noise which affects the accuracy of the results. The method proposed in this paper is to calculate approximate values for indiviudal triple lines and also define a cylindrical volume associated with the triple line.

Ideally the excess potential energy of the triple line should be decoupled from the the excess potential energies of the neighbourning grain boundaries. To minimise the effect of the grain boundaries three thin cuboid strips of atoms are selected from the triple line.  


As the radius of the cylindrical region increases the curve describing the excess potential energy is likely to assymptotically approach a straight line. The gradient of this straight line represents the total rate of change of potential energy per unit length of all three grain boundaries. 

This approach assumes that the region of interest has no defects other than the triple line and connected grain boundaries. There is also an implicit assumption that there is little spatial variation in the interfacial excess energy of each grain boundary. The grain boundaries are also assumed to be planes whose normal vector is perpendicular a fixed vector point radially from the triple line.

These assumptions are restrictive and the results obtained should be interpreted only as a possible average of the system. Systems that are modelled using molecular dynamics tend to minimise their free energy and the calculations here only involve potential energy. However energy minimisation is likely to be achieved by minismising the interfacial surface area. If the potential energy surface for the system as completely isotropic this would be achieved by planar grain boundaries.    

The algorithm produces a set of discrete points and in general these are not atom positions. The grain boundaries are constructed as a series of line segments connecting the closest points. Each line segment is then extruded vertically to give a plane section. The following algorithm is then used to identify atoms that lie sufficiently close to the grain boundary.

The total excess energy of the system is attributed to the grain boundaries, triple lines and also the strain energy density. This final contribution is a consequence of the stress exerted by the grain boundary interfaces onto the adjacent grains. The strained lattice will have a higher potential energy per unit volume. In order to calculate the excess energy due to the grain boundaries and triple lines this excess must measured relative to the strain grain potential enery per unit volume. 

A detailed analysis showed that the strain energy in the lattices also decreased with the radial distance from the triple line junction. This affect cannot adequately be explained by the straight sections of the grain boundary. Instead this appears to be a long range effect of the triple line. An analysis of the strain field also shows that a similar effect occurs at a sharp corner of a grain boundary.
 
The long range effect means that it is no longer possible to identify a fixed energy value to the triple line without also specifying a bounding region of interest around the triple line. Mathematically the triple line effect extends to an infinite radius although the magnitude of this effect descrease rapidly with radial distance. In reality the effect is likely to bounded by the dimenions of the grains as the grain boundaries will act as a barrier to further extension of the strain field. 

It therefore appears that the triple line has two distinct effects on the surrounding grains and potential energy of the simulation cell. Firstly there is a local energy which is defined for an arbitrarily close region around the triple line. Secondly there is a long range effect that decays over the radial distance in a similar way to a line dislocation.

This long range effect may also be present in a non-smooth grain boundary. The presence of a sharp corner can also produce a long range strain effect and this must be included in the analysis.

The triple line is assumed to be aligned vertically in the frame of the simulation cell. However the grain boundaries also influence the surrounding grains which are now also have an excess energy due to being strained. In general the lattice will distribute the strain evenly over its volumne due to elastic deformation. The lattice there form has excess strain energy and so its minimum  potential energy is higher than the theoretical value predicted by the embedded atom method potential.

\section{Fitting Function}

The total excess potential energy consisting of $n$ atoms can be easily calculated by subtracting the strained potential energy of an equivalent FCC grain of exactly $n$ atoms from the total potential energy of the simulation cell. The excess energy can then be partioned into a part associated with the grain boundaries and a part associated with the triple line.

The approach employed here is to model the potential energy contained within a fixed radius of the triple line. This region forms a cylinder with the triple line as its axis of symmetry. A specific function is used to model the total potential energy as a function of the radial distance from the triple line. 

The function is a sum of two simple functions that represent the contributions of the triple line and the grain boundaries respectively. This approaches assumes a simple geometry and considers the asymptotic behaviour as the radial distances becomes arbitrarily large.

The geometry of the system assumes three grain planar grain boundaries whose normal vectors are all parallel to the $\mathbf{k}$ vector of the simulation cell. The only contributions to the excess potential energy are from the three grain boundaries and the triple line. Although the grain boundaries and triples lines are two dimension and one dimensional respectively the effect on the potential energy is assumed to be three dimensional.

The effect of the three grain boundaries is assumed to increase linearly with the radius $r$. 


The complete potential energy contained with a cylindrical volume of height $h$ and radius $r$ can then be modelled using the following equation

\[ V(r) = 2 \pi h P_{L} r^2 + 3(P_G-P_L)h w r + T_0  
\]

for radius $r_c < r < r_L$ where $r_c$ is a postive critical radius.

The first term represents the total potential energy present if there were no grain boundaries or triple lines but the grains were still strained. The second term is then the excess energy due to the three grain boundaries measured above the strained grains potential energy. Finally the last term is the contribution from the triple line. Here is has been assumed that the triple line has a volume given by $\pi r_c h$ and for $r > r_c$ the triple line energy is effectively constant.

\subsection{Methodology}

The detection of grain boundaries and triple lines is essential before potential energy can be partioned into regions of interest. The position of the triple line is complicated as there no general solution to the intersection of three arbitrary planes. In general any two non-parallel planes will intersect along a line. The third plane will not normally also align such that it intersects the two planes along the same line.

The approach adopted in this paper is to extend each grain boundary as far as possible and to define the triple line as a vertical line that goes through a unique point which is equidistance from each grain boundary. 

A cylindrical region with the triple line lying along its axis of symmetry is necessarily influenced by the excess energy of the adjacent grain boundaries. This causes fluctuations in the potential energy which makes it difficult to accurately determine the energy of the triple line.


\subsection{Lattice Orientation}

The identifying of grains using their local lattice orientation cannot be used to rigorously separate different grains. Adjacent grains can have identical orientations and yet still form a grain boundary due to translation difference. Provided the translation is not a symmetry of the lattice a grain boundary can form due to a stacking difference.


\subsection{Triple Line Formation Energy}

The calculations based upon the simulation results do confirm that all the triple lines tested have a negative formation energy. The calculation was based upon the potential energies and does not include an entropic contribution. The process for calculating the theoretical formation energy of the triple line is based upon a mass balance and mean values of potential energies in the triple line, adjacent grain boundaries and the lattice.

The formation energy of the triple line is formally defined as the difference in energy of two systems with an identical number of atoms one which contains a triple line and three grain boundaries and the other contains three non-intersecting grain boundaries. The energy can be thought of the work done in changing from one system to the other.

\subsubsection{Lower Bound for Formation Energy}

Let $n_{TJ}$ be the number of triple line atoms and $U_{TJ}$ be the mean potential energy per atom. Furthermore if $U_{L}$ and $U_{GB}$ are the mean potential energies per atom for the lattice and grain boundary atoms the following expression defines the energy of formation of the triple line $E_{f}$

\[ E_{f} = n_{TJ}\left(U_{TJ} -3U_{GB} + 2U_{L} \right) \]

The term $U_{GB}$ is the averaged potential energy per atom where the summation is over all the atoms in the adjacent grain boundaries but excludes atoms defined to be in the triple line region. The equation can be interpreted as part of one grain boundary being transformed to triple line atoms whilst the other two     

\subsubsection{Upper Bound for Formation Energy}

An upper bound for the formation energy for the triple line can be estimated by assuming that the only change in energy is that the  $n_{TJ}$ atoms formed in the triple line all come from the grain boundaries and there are is no increase in the number of lattice atoms. This means that there is no increase in the number of low energy lattice atoms. In general this will be a positive quantity as the triple line atoms are higher energy than the grain boundaries. 

However this formation energy definition includes the extension  

\subsection{Verifying Numerical Results}

The simulation cell is periodic in nature and this can be used to test some of the numerical results. In particular there are four periodically equivalent triple lines positioned at the four corners of the simulation cell. The total excess energy and how it is proportioned should be very similar for each periodically equivalent triple line. It is important to emphasize that in reality there are not four distinct triple lines and they correspond to a single triple line in the periodic simluation cell.


\[ f(x) = \frac{p(x)}{q(x)} \]

with $p(0)=0$ and $\liminf{x}{f^{\prime}(x)}=0$

\[ \frac{p^{\prime}(x)q(x) -q^{\prime}(x)p(x)}{[q(x)]^2} = m\]


\[ f(x) = mx - n\ln(x+n/m)\]
\[ f'(x) = m -\frac{n}{x+p} = 0 \Rightarrow  mx + mp - n = 0 \]


\subsection{The [111] Direction}
The simulation was run with the each grains $[111]$ direction orientated parallel to the simulation cells $\mathbf{k}$ vector. An important consequence of this is that the grains now have order six rotation symmetry around the $\mathbf{k}$ vector. As the grains are also defined by regular hexagonal boundaries it is likely that grain boundary normal direction will not affect the excess potential energy as much as the previous grain configuration.

The symmetry of the simulation cell and the three grains now suggest that each triple line has a very similar neighbourhood in terms of the adjacent grains and grain boundaries. This in turn suggests that the excess energy associated with each triple line in the simulation cell should be approximately equal.

\subsection{Nucleating New Grains}

If the triple lines do have a negative excess energy compared with the grain boundaries then it is possible that there is a sufficiently strong driving force to subdivide triple lines into three and nucleate a new grain. In general there would be an energy barrier to this process and it is likely that it could only occur if the new created grain boundaries consisted of a low interfacial energy.

With the $[1,1,1]$ direction of each the lattice of each grain paralell to the $\mathbf{k}$ direction of the simulation cell this phenomenon was observed when grains 1 and 2 were rotated about $\mathbf{k}$  by an angle of 20 and 40 degrees respectively. The fact that the grain boundaries and triple lines changed significantly suggests that the initial configuration was unstable. 

However this was more than simply a reconfiguration of the existing grain boundaries and triple lines. A new grain was produced and more grain boundaries and triple lines were created. If all the grains were defect free then each would have an equal Gibbs free energy per unit volume. Grain boundaries always have an positive excess energy and so for this to be energetically favourable the triple lines should have a negative excess energy.

The pre-existing grains shown in the following image have clearly developed some defects and so the argument used in the previous paragraph is more complicated. It may still be possible that the driving force for the new grain could be accounted for by the difference in Gibbs free energy per unit volume between the grains with defects and a pristine new grain. If this were true it would this would not necessarily indicate that a triple line has a negative excess energy.



\subsection{Future Work}


Many of the triple lines studied had a positive mean excess potential energy per atom compared to the mean potential energy per atom of the three adjacent grain boundaries. This suggests that a triple line could be a good candidate for nucleating new grains. If a new grain was formed at producing three lower energy triple lines and three low energy grain boundaries then this could achieve a reduction in the overall potential energy of the system.

Two adjacent triple lines with high potential energy joined by a high energy grain boundary may also merge by effectively removing the high energy grain boundary. 


The method so far has focussed on molecular dynamics which has essentially used classical mechanics to simulate the motion of atoms based upon a emperical potential. The entropy of the system has not been considered. The energy calculations have focussed on the potential energies of different regions in the system. Molecular systems actually minimise their free energy and not necessarily the potential energy 

A future examination of the triple lines will extend this approach using density function theory. This is a well established quantum mechanical approximation and the results can be compared to the classical simulations. In general it will not be possible to run DFT simulations with the same size of simulation cell.  
\bibliographystyle{ieeetr}
\bibliography{FirstYearReport} 
\end{document}